%!TEX program = xelatex
\documentclass[11pt,numbers=noenddot]{scrartcl}
\usepackage[ngerman]{babel}
\usepackage[a4paper,lmargin={2.5cm},rmargin={3.5cm}, tmargin={2.5cm},bmargin = {2.5cm}]{geometry}
\usepackage{amsmath}
\usepackage{mathabx}
\usepackage{mathtools}
\usepackage{stmaryrd}
\usepackage{enumitem}


\usepackage{relsize}
% \linespread{1.2}
\usepackage{setspace}
\onehalfspacing
% \usepackage[square]{natbib}
\usepackage{jurabib}
% \usepackage {algorithm2e}

% Package, das die Benutzung von Old Standard erlaubt
\usepackage{fontspec}

\setmainfont{OldStandard-Regular.otf}[
Path = /usr/local/texlive/texmf-local/opentype/,
BoldFont = OldStandard-Bold.otf,
ItalicFont = OldStandard-Italic.otf]

\bibliographystyle{jurabib}
\renewcommand*{\bibbtsep}{In: }
\renewcommand*{\bibjtsep}{In: }
\jurabibsetup{
  authorformat=and
}

\makeatletter
% \renewcommand{\l@section}{\@dottedtocline{1}{1.5em}{2.6em}}
\renewcommand{\l@subsection}{\@dottedtocline{2}{4.0em}{3.6em}}
\renewcommand{\l@subsubsection}{\@dottedtocline{3}{7.4em}{4.5em}}
\makeatother

% \addtokomafont{disposition}{\normalfont}
\addtokomafont{sectionentry}{\normalfont\bfseries}


\setkomafont{subject}{\normalfont\small}
\addtokomafont{title}{\normalfont\bfseries}
\addtokomafont{section}{\normalfont\centering\bfseries}
\addtokomafont{subsection}{\normalfont\centering\bfseries}
\addtokomafont{publishers}{\normalfont\small}
\addtokomafont{date}{\normalfont\small}
\addtokomafont{author}{\normalfont\small}
\addtokomafont{descriptionlabel}{\normalfont\bfseries}

% \renewcommand{\thesection}{\Roman{section}} 
% \renewcommand{\thesubsection}{\thesection.\Roman{subsection}}


\subject{  Universität zu Köln \\
  Sprachliche Informationsverarbeitung \\
  Hauptseminar: Angewandte linguistische Datenverarbeitung \\
  Prof. Dr. Jürgen Rolshoven \\
  Hausarbeit
  }
\title{Semantische Spezifität \\im Word Space Model}
\author{Von C. Friedrich}
\date{(Vorgelegt am \today)}




\begin{document}
\begin{titlepage}
\maketitle

\abstract{Ladida, ladidu}


\thispagestyle{empty}
\end{titlepage}



\tableofcontents
\newpage

\section{Einleitung}
\subsection*{Semantische Spezifität}

\citet[11]{sparckjones1972} Beschreibt die Spezifität eines Begriffes so:
\begin{quote}
  Specificity ... is a semantic property of index terms: a term is more or less specific as its meaning is more or less detailed and precise.
\end{quote}

\subsection*{Semantische Spezifität und Anglizismen}

\section{Ein Maß für die semantische Spezifität}

\subsection{Textgrundlage}

\subsection{Document Frequency}
Bereits \citet{sparckjones1972} schlug ein statistisches Maß für die semantische Spezifität eines Wortes vor. Es ist die simple Frequenz, mit der ein Wort im Korpus auftaucht, das ein Indiz für die Spezifität darstellen soll. \citet{Caraballo99determiningthe} konnten die \emph{document frequency} als Eigenschaft von Worten dazu nutzen, für beliebige Wortpaare festzustellen, welches Wort  spezifischer oder genereller ist. Überprüft wurde das mit Beispielwortpaaren, die in einer Hyperonym- bzw. Hyponymrelation zueinander stehen: Das Wort \emph{Getränk} ist ein Oberbegriff zum Wort \emph{Cola}. Für diese Art von Relation gilt: Wenn ein Wort ein Oberbegriff eines anderen ist, so ist der Unterbegriff semantisch spezifischer als der Oberbegriff. Die klar unterschiedene Spezifität ist also eine notwendige Bedingung für die Hyperonym- bzw. Hyponymrelation. Das macht solche Wortpaare zu natürlichen Kandidaten, um Maße für semantische Spezifität zu testen.


\subsection{Satzkontexte vs. Kontextfenster}

\subsection{Größe des paradigmatischen Kontexts}

\subsection{Semantischen Nähe des Kontextes}

\subsection{Frequenz vs. Chi Squared vs. Dice Coefficient}

\subsection{Getestete Maße}

\subsection{Wordpaare}

\subsection{Resultate}

\section{Ein Modell für Anglizismen}

\subsection{Textgrundlage}

\subsection{Verwendete Technologien}

\subsection{Resultate}

\section{Konklusion}

\nocite{han2011}
\nocite{heyer2008}
\nocite{manning1999}

\bibliography{spinfohausarbeit}

\end{document}